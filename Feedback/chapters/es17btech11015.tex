Consider a feedback transconductance amplifier as shown in the figure. It utilizes an op amp with open-loop gain $\mu$, very large resistance, and an NMOS transistor $Q$. The amplifier delivers its output current to $R_L$. The feedback network, composed of resistor $R$, senses the equal current in the source terminal of $Q$ and delivers a propotional voltage $V_f$  to the negative input terminal of the op amp.

\begin{figure}[ht!]
	\begin{center}
	    %circuit
	    \resizebox{\columnwidth/1}{!}{\begin{circuitikz}[american]
\draw (5,2) node[nmos](Q){};
\draw (Q.center) node[right]{{Q}};
\draw (Q.D) (5,3.25) node[label={right:$\downarrow I_o$}]{};
\draw (Q.D) -- (5,3) to[R,l_=$R_D$] (5,4.5) node[vcc](VCC){} ;
\draw (Q.S) -- (5,0.5)node[label={right:$V_f$}]{} to[R,l_=$R$] (5,-1) node[ground](GND){};
\draw (2,2)  node[op amp, yscale=-1] (OA){};
\draw (OA.+) (0.5,2)node[label={above:$+$}]{};
\draw (OA.+) -- (-1,2.5) to[V, l= $V_{s}$] (-1,0) node[ground](GND){};
\draw (OA.-) (0.5,1.25) node[label={above:$-$}]{}; 
\draw (OA.-) -- (0.5,1.5) node[label={above:$V_i$}]{} to node[](){} (0.5,0.5) to node[](){} (5,0.5);
\draw (OA.out) to (Q.G);
\end{circuitikz}

}
	\end{center}
	\caption{}
	\label{fig:es17btech11015_fige}
\end{figure}

\begin{enumerate}[label=\arabic*.,ref=\theenumi]
\item Show that the feedback is negative.
\item Open the feedback loop by breaking the connection of $R$ to the negative input of the op amp and grounding the negative input terminal. Find an expression for $A \equiv \frac{I_o}{V_i}$
\item Find an expression for $\beta \equiv \frac{V_f}{I_o}$
\item Find an expression for $A_f \equiv \frac{I_o}{V_s}$
\item What is the condition to obtain $I_o\approx\frac{V_s}{R}$
\end{enumerate}\newpage
\begin{enumerate}[label=\arabic*.,ref=\theenumi]
\numberwithin{equation}{enumi}
\item Show that the feedback is negative.\\
\solution 
Suppose if source voltage $V_s$ increases, then $V_G$ increases\\
Since,\\
\begin{align}
I_o = \frac{1}{2}\mu_nC_{ox}\frac{W}{L}\brak{V_{GS}-V_{TH}}^2\\
    I_o = \frac{1}{2}\mu_nC_{ox}\frac{W}{L}\brak{V_G-V_f-V_{TH}}^2
\end{align}
which thereby increases $I_o$\\

Now, since $I_o$ is increasing, \\
$V_f\brak{=I_oR}$ increases.\\
and $V_f$ is feedback to negative terminal of the amplifier.\\
Hence, the system is in negative feedback.\\
\\

\item Find an expression for $A \equiv \frac{I_o}{V_i}$\\
\solution  Applying small signal analysis, we get the resultant circuit as follows 
\begin{figure}[ht!]
	\begin{center}
	    %circuit
	    \resizebox{\columnwidth/1}{!}{\begin{circuitikz}[american]
    \draw (3,2) to[isource, l=$g_mV_{GS}$] (1,2) node[label={above:$V_f$}]{};
    \draw (1,2) to(0,2) node[label=$-$]{} node[label={left:$S$}]{} ;
    \draw (3,2) to[R, l=$R_L$] (3,0) node[ground](GND){};
    \draw (0,2) to[R, l=$R$] (0,0) node[ground](GND){};
    \draw (-1.5,2) node[label=$+$]{} node[label={right:$G$}]{} to (-3,2) node[label={above:$\mu V_i$}]{};
\end{circuitikz}}
	\end{center}
	\caption{}
	\label{fig:es17btech11015_fig2}
\end{figure}
\begin{align}
    V_f = I_oR
\end{align}
and 
\begin{align}
    I_o = g_m\brak{\mu V_i-V_f}\\
= g_m\brak{\mu V_i-I_oR}\\
\implies \brak{1+g_mR}I_o = g_m\mu V_i
\end{align}
Therefore,\begin{align}
   A \equiv \frac{I_o}{V_i} = \frac{g_m\mu}{1+g_mR}
\end{align}\\

\item Find an expression for $\beta \equiv \frac{V_f}{I_o}$
\\
\solution 
From the circuit diagram, Fig: \ref{fig:es17btech11015_fige}
\begin{align}
    V_f = I_oR
\end{align}
Therefore,\begin{align}
    \beta \equiv \frac{V_f}{I_o} = R
\end{align}\\

\item Find an expression for $A_f \equiv \frac{I_o}{V_s}$
\\
\solution  Applying small signal analysis, we get the resultant circuit as follows 
\begin{figure}[ht!]
	\begin{center}
	    %circuit
	    \resizebox{\columnwidth/1}{!}{\begin{circuitikz}[american]
    \draw (3,2) to[isource, l=$g_mV_{GS}$] (1,2) node[label={above:$V_f$}]{}; 
    \draw (1,2) to(0,2) node[label=$-$]{} node[label={left:$S$}]{};
    \draw (3,2) to[R, l=$R_L$] (3,0) node[ground](GND){};
    \draw (0,2) to[R, l=$R$] (0,0) node[ground](GND){};
    \draw (-1.5,2) node[label=$+$]{} node[label={right:$G$}]{} to (-3,2) node[label={above:$\mu\brak{V_s-V_f}$}]{};
\end{circuitikz}}
	\end{center}
	\caption{}
	\label{fig:es17btech11015_fig2}
\end{figure}\\
\begin{align}
I_o = g_mV_{gs}\\
= g_m\brak{\mu\brak{V_s-V_f}-V_f}\\
= g_m\brak{\mu V_s-\brak{\mu+1}V_f}\\
= g_m\mu V_s-g_m\brak{\mu+1}I_oR\\
\implies \brak{1+g_m\brak{\mu+1}R}I_o = g_m\mu V_s\\
\end{align}
Therefore,
\begin{align}
    A_f \equiv \frac{I_o}{V_s} = \frac{g_m\mu}{1+g_m\brak{\mu+1}R}
\end{align}

\item What is the condition to obtain $I_o\approx\frac{V_s}{R}$\\
\solution
From the circuit diagram, Fig: \ref{fig:es17btech11015_fige}
\begin{align}
    I_o = \frac{1}{2}\mu_nC_{ox}\frac{W}{L}\brak{V_{GS}-V_{TH}}^2\\
    I_o = \frac{1}{2}\mu_nC_{ox}\frac{W}{L}\brak{V_G-V_f-V_{TH}}^2
\end{align}
and \begin{align}
    V_f = I_oR
\end{align}
\begin{align}
    \implies I_o = \frac{1}{2}\mu_nC_{ox}\frac{W}{L}\brak{\mu V_s-\brak{\mu+1}I_oR-V_{TH}}^2
\label{eq:es17btech11015_eq1}    
\end{align}
If\begin{align}
    I_o << \frac{\mu_nC_{ox}W}{2L}
\end{align} and
\begin{align}
    \mu >> 1
\end{align}
From equation \eqref{eq:es17btech11015_eq1}, we get
\begin{align}
    \sqrt{\frac{2I_o}{\mu_nC_{ox}\frac{W}{L}}} << \mu V_s-\brak{\mu+1}I_oR\\
\end{align}
then,\begin{align}
    \mu V_s \approx \brak{\mu+1}I_oR-V_{TH}
\end{align}
which can be approximated to 
\begin{align}
    V_s \approx I_oR
\end{align}
Thus, $\mu>>1$ and $I_o<<\frac{\mu_nC_{ox}W}{2L}$ are the conditions to obtain $V_s \approx I_oR$
% 
\end{enumerate}



